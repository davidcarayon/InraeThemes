
\thispagestyle{empty}
\newgeometry{left=2cm,bottom=0.1cm}
\vspace*{4cm}
\setlength{\parindent}{0cm}
\textcolor{inrae}{\sffamily\Huge\bfseries\titre}\par\bigskip
\textcolor{inrae}{\sffamily\dateVersion{} -- \version}\par

% Sigle INRAE
\vspace*{3cm}
\includegraphics{inrae}\par\bigskip
\textcolor{inrae}{\sffamily\Large Institut national de recherche pour}\par
\textcolor{inrae}{\sffamily\Large\bfseries l'agriculture, l'alimentation et l'environnement}\par\bigskip

% Logo INRAE

\vspace*{2cm}
\hspace{-5cm}
{\transparent{0.4}\includegraphics[width=10cm]{sigle-inrae-plein}}

\vspace*{2cm}
\textcolor{inrae}{\sffamily\auteur}\par

\textcolor{inrae}{\sffamily
Document distribué sous licence CC-BY}\par
  \includegraphics[width=1cm]{cc-by} \href{https://creativecommons.org/licenses/by/4.0/fr/legalcode}{https://creativecommons.org/licenses/by/4.0/fr/legalcode}
  
\restoregeometry
