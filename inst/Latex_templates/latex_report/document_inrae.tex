\documentclass[12pt
,a4paper
,titlepage
,twoside
%,openany % ouvre les sections indifféremment sur la page de gauche ou de droite
]{book}

% Définition des marges de la page
\usepackage[left=3cm,right=3cm,top=2cm,bottom=2cm]{geometry}

% Paramétrages de la langue et de l'encodage
\usepackage[utf8x]{inputenc}
\usepackage[square,sort,comma,numbers]{natbib}
\usepackage[french]{babel}
\usepackage[T1]{fontenc}

%\usepackage{fontspec}
%\setmainfont{Linotype-AvenirNextLTPro.ttf}[
%BoldFont = Linotype-AvenirNextLTProBold.ttf,
%ItalicFont = Linotype-AvenirNextLTProItalic.ttf,
%BoldItalicFont = fLinotype-AvenirNextLTProBoldItalic.ttf
%]

% formules mathématiques
\usepackage{amsmath}
\usepackage{amsfonts}
\usepackage{amssymb}

% gestion des graphiques
\usepackage{graphicx}
\usepackage{transparent}

% Gestion des couleurs
\usepackage[usenames,dvipsnames]{xcolor}
\definecolor{inrae}{RGB}{0,163,166} % inrae
\definecolor{inraeclair}{RGB}{102,193,191}   % inrae clair
\definecolor{inraemedium}{RGB}{0,140,142}   % inrae medium
\definecolor{inraefonce}{RGB}{39,86, 98} % inrae foncé
\definecolor{vert}{RGB}{157,197,68} % vert
\definecolor{bleuclair}{RGB}{158,214,227} % bleu clair
\definecolor{bleufonce}{RGB}{66,48,137} % bleu foncé
\definecolor{gris}{RGB}{121,120,112}   % gris
\definecolor{argent}{RGB}{196,192,179} % argent
\definecolor{rouge}{RGB}{142,2,0} % rouge

% Redefinition des liens web
\usepackage{url}
\usepackage{hyperref}
\hypersetup{
urlcolor=inraefonce
,linkcolor=.
,colorlinks=true}


% Polices de caractères
\usepackage{times}
\usepackage{mathptmx} % times, y compris dans les formules mathématiques
\renewcommand{\familydefault}{\sfdefault}

% Insertion de code source dans le texte, à utiliser avec \begin{lstlisting} et \lstset{java|html|php...}
\usepackage{listings}

% Définition des entêtes
\usepackage{fancyhdr}
\pagestyle{fancy}

% Redéfinition des titres de section
\usepackage{titlesec}

% Insertion de graphiques à des emplacements définis
\usepackage[abs]{overpic}

% règles typographiques de l'Imprimerie nationale
\usepackage[all]{nowidow}
\usepackage[
% frenchchapters renomme le premier chapitre, mais :
%	- cela pose problème dans la table des matières
%	- cela ne peut être utilisé qu'avec la renumérotation des chapitres activée
%frenchchapters,
parindent,
lastparline,
hyphenation
]{impnattypo}

% Renumérotation des chapitres
%\renewcommand{\thesection}{\Alph{section})}
%\renewcommand{\thesubsection}{\arabic{subsection} -}
%\renewcommand{\thesubsubsection}{\alph{subsubsection} -}
%\usepackage{engrec}
%\renewcommand{\theparagraph}{\engrec{paragraph})}
%\setcounter{secnumdepth}{4}

% Gestion de la page blanche avant la 4ème de couverture 
\usepackage{scrextend}

% Génération du code Ipsum lorem
\usepackage{blindtext}


% Definition des chapitres
%\usepackage{sectsty}
\titleformat{\chapter}[display]
{\normalfont\Huge\filcenter\sffamily\color{inrae}}
{\Large{\chaptertitlename~\thechapter}}
 {1em}{}

%{\titlerule
%% \vspace{1pc}%


\titleformat{\section}
{\color{inrae}\normalfont\Large\bfseries\sffamily}
{\color{inrae}\thesection}{1em}{}

\titleformat{\subsection}
{\color{inraefonce}\bfseries\sffamily}
{\color{inraefonce}\thesubsection}{1em}{}

% Bibliographie
% natbib est indispensable si la biblio contient des accents
% Options pour natbib (extrait de http://merkel.zoneo.net/Latex/natbib.php)
%    round: (par défaut) pour des parenthèses arondies (());
%    square: pour des crochets ([]);
%    curly: pour des accolades ({});
%    angle: pour des équerres (<>) ;
%    colon: (par défaut) pour séparer les citations multiples par deux points (:);
%    comma: pour utiliser une virgule comme séparateur;
%    authoryear: (par défaut) pour des citations auteurs-année;
%    numbers: pour des citations numériques;
%    super: pour des citations numériques en exposant, comme dans Nature;
%    sort: ordonne les citations multiples dans l'ordre dans lequel elles apparaissent dans la bibliographie;
%    sort&compress: comme sort mais en plus les citations numériques multiples sont comprimées, si possible (3-6, 15, par exemple);
%    longnamesfirst: transforme la première citation à une référence en une version étoilée (avec la liste complète des auteurs) et le citations suivantes normales (liste abbrégée);
%    sectionbib: pour redéfinir \thebibliography pour avoir une \section* à la place d'un \chapter*; valide seulement pour les classes de document possédant la commande \chapter; à utiliser avec le paquetage  chapterbib;
%    nonamebreak: garde tous les noms d'auteurs d'une citation sur une même ligne; celà cause des problèmes de débordement, mais permet de résoudre certains problèmes liés à hyperref.

% En cas de souci, supprimez les fichiers .aux et .bbi après modification des paramètres
% styles natbib natifs : abbrvnat, plainnat, unsrtnat
\bibliographystyle{unsrtnat}
\usepackage{hypernat}
% Ajout de la référence à la bibliographie dans la table des matières
\usepackage[nottoc, notlof, notlot]{tocbibind}

%Données de titre et d'auteur pour la page de garde
\newcommand{\titre}{Titre du document}
\newcommand{\sousTitre}{Sous-titre du document}
\newcommand{\auteur}{auteur}
\newcommand{\dateVersion}{20 décembre 2019}
\newcommand{\version}{Version 1}
% Limite le sommaire à la section (supprime les sous-sections et en dessous)
\setcounter{tocdepth}{1}

\begin{document}

%Supprime les veuves et orphelines
\widowpenalty=10000
\clubpenalty=10000
\raggedbottom 

% Integrer la page de garde

\thispagestyle{empty}
\newgeometry{left=2cm,bottom=0.1cm}
\vspace*{4cm}
\setlength{\parindent}{0cm}
\textcolor{inrae}{\sffamily\Huge\bfseries\titre}\par\bigskip
\textcolor{inrae}{\sffamily\dateVersion{} -- \version}\par

% Sigle INRAE
\vspace*{3cm}
\includegraphics{inrae}\par\bigskip
\textcolor{inrae}{\sffamily\Large Institut national de recherche pour}\par
\textcolor{inrae}{\sffamily\Large\bfseries l'agriculture, l'alimentation et l'environnement}\par\bigskip

% Logo INRAE

\vspace*{2cm}
\hspace{-5cm}
{\transparent{0.4}\includegraphics[width=10cm]{sigle-inrae-plein}}

\vspace*{2cm}
\textcolor{inrae}{\sffamily\auteur}\par

\textcolor{inrae}{\sffamily
Document distribué sous licence CC-BY}\par
  \includegraphics[width=1cm]{cc-by} \href{https://creativecommons.org/licenses/by/4.0/fr/legalcode}{https://creativecommons.org/licenses/by/4.0/fr/legalcode}
  
\restoregeometry

\setlength{\parindent}{0.5cm}
% Définition des entêtes
\fancyhead{}
\renewcommand{\headrulewidth}{0pt}
\fancyhead[CO]{\color{inrae}\nouppercase\leftmark}
\fancyhead[CE]{\color{inrae}\titre{}}
\fancyfoot[C]{\thepage}
% Redéfinition de \cleardoublepage pour créer une page totalement vide
\makeatletter
\def\cleardoublepage{\clearpage\if@twoside \ifodd\c@page\else
  \hbox{}
  \vspace*{\fill}

  \vspace{\fill}
  \thispagestyle{empty}
  \newpage
  \if@twocolumn\hbox{}\newpage\fi\fi\fi}
\makeatother

% \cleardoublepage permet de générer une page vide 
% si le chapitre ne commence pas sur la page de droite
\renewcommand*\contentsname{Sommaire}
\cleardoublepage
% Table des matières
\tableofcontents

% Ajout d'un préambule
\frontmatter
\cleardoublepage
\chapter{Préambule}
\blindtext

% Début réel du texte
\mainmatter
%\cleardoublepage
\chapter{Introduction}
\section{Première section}
Référence bibliographique au livre \citet{livre}, à l'article \citet{article}, à la thèse \citep{these}.

La bibliographie utilise le fichier \textit{irstea.bib} dans cet exemple.
\subsection{sous-section}
\blindtext
\subsection{autre sous-section}
\blindtext

\blindtext

\blindtext

% Second chapitre
%\cleardoublepage
\chapter{Second chapitre}
% \input{chapitre2}

%Annexes
\appendix
\chapter{Première annexe}

%Bibliographie
\backmatter
% Integration de la biblio
% Pour insérer toutes les références : 
%\nocite{*}
% Pour intégrer une référence non citée : 
%\nocite{ref}
\nocite{article_non_cite}
\bibliography{inrae}

% 4ème de couverture
\ifthispageodd{\hbox{}
  \vspace*{\fill}
  \vspace{\fill}
  \thispagestyle{empty}\newpage }{}
% Template basé sur le code d'E. Quinton (INRAE)

\thispagestyle{empty}

\newgeometry{left=2cm,bottom=0.1cm}

\begin{center}

\color{inrae}

\vspace*{10cm}

\includegraphics[height=0.6cm]{templates/fleche-titre}\par

\sffamily
UR -- Unité de recherche\\
Adresse\\
Code Postal COMMUNE\\
+33 (0)0 00 00 00 00\par\bigskip

Rejoignez-nous sur :\\
\includegraphics{templates/reseaux-sociaux}\par\bigskip

\vspace*{2cm}

{\bfseries Institut national de recherche pour\\
l'agriculture, l'alimentation et l'environnement}\par\bigskip

\includegraphics[width=5cm]{templates/bloc-etat}\par

\end{center}

\restoregeometry

\end{document}
