\documentclass[11pt,aspectratio=169]{beamer}
% pour fonctionner, ce modele doit être accompagné de
% beamercolorthemeinrae.sty
% inrae.png
% fleche-titre.png
% sigle-inrae.png
% sigle-inrae-plein.png
% et, pour la licence Creative commons : cc-by.png
%\usecolortheme[secheader]{inrae} % Version avec barre de navigation supérieure
\usecolortheme[default]{inrae} % Version sans barre de navigation supérieure
\usepackage[utf8]{inputenc}
\usepackage[french]{babel}
\usepackage[T1]{fontenc}
\usepackage{amsmath}
\usepackage{amsfonts}
\usepackage{amssymb}
\usepackage{graphicx}
\usepackage{url}
\usepackage{xcolor}
\usepackage{wrapfig}
\usepackage{textpos}
\usepackage{array}
\usepackage{color}
\usepackage{transparent}
\usepackage{hyperref}
\hypersetup{
urlcolor=inraefonce
,linkcolor=.
,colorlinks=true}

% Definition de la page de titre
\defbeamertemplate*{title page}{customized}[1][]
{
  \hspace{1.6cm}\includegraphics[width=2.5cm]{Republique_Francaise_RVB}\vskip -1.6cm
  \hspace{5cm}\includegraphics[width=2cm]{inrae}\par\bigskip
  \hspace{1.2cm}\includegraphics[height=0.8cm]{fleche-titre} \usebeamerfont{title}\textcolor{inrae}{\inserttitle}\par
  \hspace{2cm}\usebeamerfont{subtitle}\usebeamercolor[fg]{subtitle}\insertsubtitle\par
  \bigskip
  \hspace{2cm}\usebeamerfont{author}\insertauthor\par\bigskip
  \hspace{2cm}\usebeamerfont{date}\usebeamercolor[fg]{date}\insertdate\par\bigskip \hspace{2cm}\usebeamerfont{institute}\insertinstitute\par
  {\small
  \hspace{2cm}Document distribué sous licence CC-BY\\
  \hspace{2cm}\includegraphics[width=1cm]{cc-by} \href{https://creativecommons.org/licenses/by/4.0/fr/legalcode}{https://creativecommons.org/licenses/by/4.0/fr/legalcode}
  }\par
  \vspace{-5.7cm}
  \hspace{-0.5cm}
  \includegraphics[width=6cm]{sigle-inrae}
}

% Données générales
\author[Auteur]{Auteur--
\href{mailto:auteur@inrae.fr}{auteur@inrae.fr}
}

\title[Titre court]{Titre de la présentation}
\subtitle{sous-titre}
\institute[UR]{UR -- Déroulé acronyme}
\date[\today]{\today}

\begin{document}

% Page de titre
\begin{frame}[plain]
\titlepage
\end{frame}

% Page de sommaire
\begin{frame}[plain]
\tableofcontents
\end{frame}


\section{Première partie}
% Diapo pour identifier une nouvelle section
\setbeamercolor{background canvas}{bg=inraeclair}
\begin{frame}[plain]{}
	\begin {center}
		\textcolor{inraefonce}{\textbf{{\Large Première partie}}}
	\end{center}
\end{frame}
\setbeamercolor{background canvas}{bg=white}
% fin de la diapo d'identification d'une nouvelle section

\begin{frame}{Titre diapo}{Sous-titre diapo}

\end{frame}


\begin{frame}{une diapo présentant les boites}
\begin{block}{Les points à élucider...}
Quels sont les points à élucider ?
\end{block}

\begin{exampleblock}{Un exemple}
Voici un exemple...
\end{exampleblock}

\begin{alertblock}{Attention...}
Faites attention à :
\begin{itemize}
\item ceci...
\item ... et cela.
\end{itemize}
\end{alertblock}
\end{frame}

\begin{frame}{Un exemple d'affichage décalé}
\begin{enumerate}
\item<1-> section 2

\item<2-> La suite de la section 2

\begin{itemize}
\item<3> en retrait de la section 2
\item<3> toujours en retrait de la section 2
\begin{itemize}
	\item<3> en sous-retrait
\end{itemize}
\end{itemize}

\item<2-> et la fin de la section 2...
\item<4-> Les alinéas ont disparu !
\end{enumerate}

\end{frame}


\end{document}

